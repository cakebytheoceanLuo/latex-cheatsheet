\section{Beweistechniken}
\subsection*{Direkter Beweis}
Beim direkten Beweis wird Schritt für Schritt mittels \emph{Wenn, Dann} bewiesen.
\subsection*{Kontraposition}
Da $p\Rightarrow q\equiv \neg q\Rightarrow \neg p$ kann man die Aussage auch mittels Kontraposition beweisen.
\subsection*{Widerspruch}
Beim Widerspruchsbeweis wird Gegenteil angenommen und in einen Widerspruch geführt.
Also muss die ursprüngliche Aussage wahr sein.
\subsection*{Äquivalenzbeweis}
Beweis über zeigen der Hin- und Rückrichtung.
\subsection*{Fallunterscheidung}
Beweis aller möglichen Fälle.
\subsection*{Induktionsbeweis}
Induktionsanfang ($n$ kleinste Zahl):\\
Induktionsbehauptung: Aussage gelte für beliebiges aber festes $n\in \mathbb{N}$ mit $n\geq$ kleinste Zahl.\\
Induktionsschluss ($n \Longrightarrow n+1$): Zu zeigen ist also $n+1$ einsetzen $\Rightarrow$ Aussage gilt auch,
\emph{mit Benutzung von Induktionsbehauptung}.
%\subsection*{VI an Rekursiver Funktion}