\section{C5-Messaging\&Queuing}
\btext{MessageQueuingPattern}data shared across applications on different platforms, MQP  transfer packets of data (messages) frequently, immediately, reliably, 
exactly once, and asynchronously, using customizable formats (Enterprise application integration EAI)
\textbar MessagePassing: message(receiver+sender+type+payload) queuing without message buffering
\lstinline{send(message); receive(message); callback();}
\textbar \btext{MessagingvsRMI}
One interface (per messaging product) with generic operations,
Lower-level of abstraction than remote invocations,
Flexible in that messaging allows arbitrary interaction patterns between sender and receiver,
Sender is not blocked after sending,
Can emulate request/reply pattern,
Asynchronous behavior more difficult to use and debug
$\Leftrightarrow$
Interface required and known, but differs across apps, 
Programming model resembles non-remote calls,
Realizes request/reply pattern,
Sender blocked until reply arrives, unless there is an error,
As compared to non-remote call, more potential for failures
\textbar \btext{MessagePassingProperties}
Reliability(SequenceNum, Ack, Timeout and re-transmission$\rightarrow$Message loss,
Check-sums, Error correcting codes (redundancy in transmission to reconstruct),Repeated sending $\rightarrow$ Message Corruption)
ordering(message arrive in a specific order: FIFO,Causal,Total),
Synchronous vs. asynchronous sending
Synchronous vs. asynchronous receiving
Buffering of messaging $\rightarrow$ Queuing
\textbar \btext{SynchronousComm}
Sender blocked until receive on receiver-side completed,
Receiver blocked until sendon sender-side completed,
Syn equalizes sender/receiver speeds $\rightarrow$ Deadlock,
\btext{+:}
Sender knows that message was received,
Only single message needs to be buffered,
Synchronizes sender and receiver operating at different speeds
\btext{-:}
Sender and receiver are coupled (i.e., need to be up at the same time),
Sender and receiver are blocked, which reduces potential for parallelism and potential for deadlock
\textbar \btext{AsyCom}
Send not blocked; MW buffers message on sender-side or on receiver-side,
SenderBuffer enables to temporarily send faster than message transmission,
Receiver buffer enables to temporarily receive more messages than can be processed,
Buffers temporarily mitigate speed variances of sender and receiver, but not permanently!
\btext{+:}
Sender and receiver are loosely coupled, not same time
More potential for parallelism than in synchronous case
\btext{-:}
Sender does not know if or when a message was received,
Buffers full,
Dev. Debug. complex
\textbar \btext{MessageBuffering}
BufferOverflow
send buffer full $\rightarrow$
Sender blocks,
A buffered message is over-written,
Exception, error returned to sender
\textbar
receive buffer full $\rightarrow$
Message dropped,
A buffered message is over-written,
A NACK is returned to MW at sender.
\textbar \btext{Sliding Window ProtocolTCP}
prevent receiver buffer to overflow:
sending side may only have a set number of unacknowledged messages (the sliding window):
Receiver sends out ack,
SW size is determined statically or dynamically
Window size < buffer size (receiver buffer never overflows),
Window size > buffer size( Potential for higher throughput but risk of buffer overflow
,Use flow control (dynamically adjusting the window size))
\textbar \rtext{Queuing}:message passing + message buffering
\textbar \btext{Decoupling Ser,Rer} $\rightarrow$ indirect and asy compared(via Callback = deque)
Space de.(Ser not know Rer, but know Queue),
Time de.(Ser Rer not same time),
Flow de.(Ser not blocked sending$\rightarrow$non-blocking enqueue)
\textbar \btext{MessageQueuingPatterns}
OneToOne, OneToMany, ManyToOne, ManyToMany
\textbar \btext{QueueManager}
Specialized component providing queuing functionality to apps
administrative functions(Creation and deletion of queues, Starting and stopping of queues,
Altering properties of existing queues,Monitoring of performance, failures, and recoveries)
, Often queue managers can be configured to forward messages to other queue managers to form a network
\textbar \btext{TransactionQueue}recovery from failures,ACID(atomic,consistent, isolated durable)
Enable local messaging transactions
, Group a set of consumed \& produced messages into an atomic unit of work
, A transaction either commits (succeeds) or aborts (fails)
, Messages are actually received/sent if transaction commits
, Producer side(Produced messages are retained until commit, If transaction aborts, messages are discarded)
, Consumer side(All consumed messages are kept until commit, If transaction aborts, messages are re-delivered)
Sending of a request and receiving a corresponding reply cannot be part of a single local messaging transaction
\textbar \btext{Request/Reply Queue,not atomic}Request with correlationID used to match with reply at C
C $\overset{1.Req.enq}{\rightarrow}$ ReqQue $\overset{2.Req.deq}{\rightarrow}$ S
$\overset{3.Req.process}{\rightarrow}$ RepQue $\overset{4.Rep.deq}{\rightarrow}$ C
\textbar \btext{Request/Reply TXQueue} local transactions:1.C RequestTX: C enqueues request.2. S TX: Server dequeues request, processes request, and enqueues reply
3. C Reply TX: C dequeues reply
\textbar C RequestTX once committed: Request processed exactly-once, Reply processed at-least-once
\textbar C ReqTX\{Start enqueueRequest Commit\} $\rightarrow$ Req.Que 
$\rightarrow$ S TX\{Start dequeueRequest enqueueReply Commit\} $\rightarrow$ Rep.Que 
$\rightarrow$ C RepTX\{Start dequeueReply Commit\}

C checks queues(Request is either in request queue or Request is currently processed by the server or Reply is in the reply queue)
, Actions taken in case of abort(If Client Request TX aborts, enqueuing of request is undone,
If Server TX aborts, dequeuing of request and enqueuing of reply is undone,
If Client Reply TX aborts, dequeuing of reply is undone)
\textbar \btext{C crash}
1. C ReqTX not commit $R \neq R_e$ $\rightarrow$ No message in any queue, old $R_e$ in QM
2. C ReqTX committed but S TX not $\rightarrow$ Req in ReqQue or being processed by S,
3. S TX committed but C RepTX not $\rightarrow$ Reply in RepQue(nonempty),
4. C RepTX committed $R \neq R_d$ $\rightarrow$ No message in any queue.successful
Persistent storage at C and queue manager required:
C marks each req with ID, QM stores IDs of the last enqueued request $R_e$
and of the last dequeued reply $R_d$, updated after commits.
R:ID of the most recent request by the C (stored by the client in local persistent storage):
C processes reply, if reply queue is non-empty (R = $R_e$ but QM’s and C's Reply IDs do not match)
, C checks reply queue for reply and waits (if necessary)
\btext{Distributed TXQueuing} Two-Phase Commit(2PC):Only atomicity

1.Prepare (to commit, voting phase)
(Each participant votes to commit or to abort the TX,
Once a participant has voted to commit, it can no longer abort the TX unilaterally)
2.Commit(completion phase)
(Participants actually commit, after consensus has been reached that all participants are prepared.
Otherwise, all participants abort)
\btext{Code} \lstinline{ChannelFactory cf = new ChannelFactory();Producer p;Consumer c;}
init():  
\lstinline{cf.setHost("broker.tum.de");cf.queueDeclare("req\_q", Persistent); cf.queueDeclare("res\_q", Persistent); p = cf.newProducer("req\_q"); c = cf.newConsumer("res\_q");//for server change p and c, C S different init()}
\btext{RRQueue} Client: 
send(byte[] msg):
\lstinline{p.enqueue(msg);}
byte[] receive():
\lstinline{return c.dequeue();}
\textbar Server: run()
\lstinline{while(true) byte[] req = c.dequeue(); try Database.store(req); p.enqueue('ACK'.toBytes()) catch(e) p.enqueue('ERROR'.toBytes());}
\btext{Correlation}
Client:
\lstinline{Hashmap<Guid, byte[]> intermediate = new Hashmap<Guid, byte[]>(); Hashmap<Guid, Correlation> correlations = new Hashmap<Guid, Correlation>();// two also for Broker}
run():
\lstinline{Thread.start(new Thread() void run() while(True) Delivery d = c.dequeue();/*at broker responseConsumer.dequeue();*/ if(delivery != null) Guid g = d.getGuid(); if(intermediate.containsKey(g) Correlation cor = new Correlation(g, intermediate.get(g),d.getMessage()); intermediate.remove(g); cor.add(g, correlation);/*at Broker clientQueues.get(guid.getClientName()).enqueue(g, cor.toBytes());*/}
send(byte[] msg):
\lstinline{Guid g = Guid.generate(); p.enqueue(g,msg); intermediate.add(g, msg);//not in Broker C}
\textbar Server: run()
\lstinline{while(true) Delivery req = c.dequeue(); try Database.store(req); p.enqueue(req.getGuid(), 'ACK'.toBytes()) catch(e) p.enqueue(req.getGuid(),'ERROR'.toBytes());}
\btext{Broker}
Client: \lstinline{String clientName;cf.queueDeclare('request_queue_broker');cf.queueDeclare('client_queue_'+clientName);}
void receive():
\lstinline{Delivery d = c.dequeue(); if(d != null) Guid g = d.getGuid(); Correlation cor = new Correlation(); cor.fromBytes(d.getMessage());}
\textbar Broker:
\lstinline{List<String> clients = new ArrayList<String>(); List<String> servers = new ArrayList<String>(); Map<String, Producer> serverQueues = new HashMap<String, Producer>(); Map<String, Producer> clientQueues = new HashMap<String, Producer>();}
init():
\lstinline{clients.add("c1"); servers.add("s1");/*multi add*/ cf.queueDeclare("request_queue_broker", Persistent); requestConsumer = cf.newConsumer("response_queue_broker"); cf.queueDeclare("response_queue_broker", Persistent); responseConsumer = cf.newConsumer("response_queue_broker"); for(String server : servers) /* also for client*/ String queueName = "server_queue_"+server; /*clinet_q*/ cf.queueDeclare(queueName, Persistent); serverQueues.put(client, cf.newProducer(queueName));/*clineQueues*/}
run():
\lstinline{Thread.start(new Thread() void run() int count = 0; while(True) Delivery d = requestConsumer.dequeue(); if(d != null) intermediate.add(d.getGuid(), d.getMessage()); serverQueues.get(count\%servers.size()).enqueue(d.getGuid(),d.getMessage()); count++; // with Func Cor Client run}
\textbar Server
\lstinline{String sName;}
run():
\lstinline{while(True) Delivery d = c.dequeue(try Database.store(request); p.enqueue(d.getGuid(), 'ACK'.toBytes()); catch(e) p.enqueue(d.getGuid(), 'ERROR'.toBytes());}
\textbar \btext{TXQueue} Client:
\lstinline{Queue<byte[]>messageBuffer = new Queue<byte[]>;Queue<Delivery> resultBuffer = new Queue<Delivery>;}
request():
\lstinline{guid = Guid.generateRandom();file = newFile('path/requestFile');file.write(guid);startTX message = messageBuffer.get(); producer.enqueue(guid, message); commitTX}
reply():
\lstinline{startTX Delivery delivery = consumer.dequeue(); resultBuffer.put(delivery); commitTX;}
recovery():
\lstinline{file = newFile("path/requestFile"); Guid R = file.readLastEntry(); Guid Re = cf.getLastEnqueued("request_queue"); Guid Rd = cf.getLastDequeued("reply_queue"); if(R != Re) requestTX() else if(R != Rd) while(cf.getQueueSize('reply_queue') == 0) Thread.sleep(500); replyTX() else /*nothing*/ spawn(while (true))}
\textbar Server: \lstinline{processTX startTX Delivery delivery = consumer.dequeue(); producer.enqueue(delivery.getCorrId(), 'ACK'); commitTX;}

