\section{Service Orchestration}
\rtext{Business Process}: set of linked activities which realise business goal
\redtext{Bus. Pro. vs Programs}
Granularity(Based on activities, Programming in the large),
Control flow(Explicitly defined, Easy understand),
Flexibility(change),
Execution(call third-party webservices, Scheduled by process engine)
\bluetext{VS}
Granularity(Based on instructions, Programming in the small)
Control flow(Implicitly defined, Not easy understand)
No Flexibility(Hard-coded)
Execution(Invoke instructions locally, Scheduled by operating system)
\rtext{BPEL, WS-BPEL}:Web Services Business Process Execution Language(XML based),
to orchestrate loosely coupled services(to model \bluetext{SOA's Business processe}),
using web services running at bus. proc. execution engine (ODE),
Lacks possibility for formal verification compared to \redtext{FSP}
\redtext{Components}:
Activities,
State handling,
Control structures,
Exception handling,
Partner links,
Parallism,
Dead path elimination
\redtext{Related Standards}
\bluetext{SOAP}
Simple Object Access Protocol,
Comm. platform in XML
\bluetext{WSDL}
Web Service Definition Language,
set of endpoints operating on messages,
Operations and messages described abstractly and bound to concrete network protocol
\bluetext{UDDI}
Universal Description, Discovery and Integration,
Services registered, published and reused by other organizations,
App store for webservices

\rtext{Service-Oriented ArchitectureSOA}
technique that involves the interaction between loosely coupled services that function independent
\redtext{Principles}:
Loose coupling,
Service contract $\rightarrow$ WSDL,
Abstraction of underlying logic,
Autonomy,
Reusability,
Composability $\rightarrow$ WS-BPEL,
Interoperability,
Discoverability $\rightarrow$ UDDI
\textbar
\btext{BPMN vs BPEL}
Business friendly,
intuitive,
process oriented,
Swimlanes that represent organizational units,
User friendly data manipulations,
include human tasks,
expose web service UI
\redtext{VS}
technical,
manipulated with XPath expressions,
xpress orchestrations,
error handling,
compensating actions
\textbar
\btext{Service composition}
\bluetext{conceptual model:} Servers exchanged Msg via out-port, in-port.
\bluetext{design methodology:}
\redtext{1Bottom-up:}
Service providers develop and publish services
Service consumers discover and select services
Service consumers compose selected services
\redtext{2Top-down:}
Service consumer develop a global process
Service consumers decompose the global process into subprocesses
Service consumers select/develop services to implement subprocesses
\textbar
\rtext{Service orchestration}
\bluetext{Local perspective},
Describe control from one party‘s behavior(perspective),
\bluetext{WS-BPEL} as Standard,
Executable processes which interact(message level) with web services
(Service internal / external),
Business processes(business logic $+$ task execution order,
span multiple apps and organizations,
Define a long-lived transactional multistep process model)
\textbar
\rtext{Service choreography}
Global perspective,
Describe the global interactions among all the parties,
\bluetext{WS-CDL} as Standard
Tracks the message sequences among multiple source and sinks,
Each involved party describes its part of the interaction
\textbar
\rtext{Orchest. vs Choreo.} Two Orchest.(Web Service) sending each other(Choreo.) \bluetext{Request, ACK, Accept, ACK}
\textbar
\rtext{Formal Methods:}
Modeling approaches, State machine verification,
(Finite state processes (FSP), BPEL $\rightarrow$ LTS $\rightarrow$ FSP),
to represent and reason about \bluetext{Complexity of Concurrency Sy},
Performance optimization,
Deadlock detection, 
dead path elimination
\textbar
\rtext{FSP}
Labeled Transition System(
Abstract machine to study computation
Contains a set of states and transitions between states)
\textbar
\rtext{BPEL $\rightarrow$ FSP}
\bluetext{Activitiy}:
<invoke partner='p1' operation='o1' /> $\rightarrow$ INVOKE = (invoke\_p1\_o1 -> END).
<receive partner='p2' operation='o2' /> $\rightarrow$ RECEIVE = (receive\_p2\_o2 -> END).
<reply partner='p1' operation='o1' />  $\rightarrow$ REPLY = (reply\_p1\_o1-> END).
\bluetext{Sequence}:
<sequence>
\begin{CJK*}{UTF8}{gbsn}
  \redtext{见上面BPEL三个}
\end{CJK*}
</sequence> 
$\rightarrow$
\begin{CJK*}{UTF8}{gbsn}
\redtext{见上面FSP三个}
\end{CJK*}
SEQUENCE = INVOKE; RECEIVE; REPLY; END. 
$0 \rightarrow 1\rightarrow 2 \rightarrow E$
\bluetext{Flow: Parallel Activity}
<flow>
\begin{CJK*}{UTF8}{gbsn}
\redtext{见上面BPEL三个}
\end{CJK*}
</flow>
$\rightarrow$
\begin{CJK*}{UTF8}{gbsn}
\redtext{见上面FSP三个}
\end{CJK*}
|| FLOW = (INVOKE || RECEIVE || REPLY).
$0 \rightarrow 1\rightarrow 2 \rightarrow E \leftarrow 4 \leftarrow 5 \leftarrow 6 \leftarrow 7$
\begin{CJK*}{UTF8}{gbsn}
\redtext{而且每点出3种Activity}
\end{CJK*}
\textbar
\rtext{Deadlock Detection}:
when two or more competing activities are each waiting for the other to finish,
\redtext{in FSP: a non-final-state with no outgoing arcs, like A sends to B, B sends to A}
\bluetext{Assumption: Sync Comm}
$!m_1$: Send a Msg of type m,
$?m_1$: Receive a Msg of type m
